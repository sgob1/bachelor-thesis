\documentclass{beamer}
\usetheme{UNIUD}
%\usecolortheme{default}
\usepackage[a-1b]{pdfx}
\usepackage[T1]{fontenc} 
\usepackage[utf8]{inputenc}
\usepackage[italian]{babel}

\title{Script Lua per il controllo remoto di droni}
\subtitle{Tesi sperimentale}
\author{Marco Sgobino}
\institute{Università degli Studi di Udine}
\date{}

%\AtBeginSection[]
%{
    %\begin{frame}
        %\frametitle{Script Lua per il controllo remoto di droni}
        %\tableofcontents[currentsection]
    %\end{frame}
%}


\begin{document}
\begin{frame}
        \titlepage
\end{frame}

\section{Introduzione}

\begin{frame}
        \frametitle{Premessa}
        In questo elaborato di tesi sperimentale sono stati effettuati:
        \begin{enumerate}
                \item lo studio della documentazione e comprensione del funzionamento di OpenTX
                \item il progetto e sviluppo di script in Lua eseguiti su firmware OpenTX
                \item la sperimentazione tramite simulatore e successivamente con radiocomando reale
        \end{enumerate}
\end{frame}

\begin{frame}
        \frametitle{Introduzione a OpenTX}
        OpenTX è un firmware per radiocomandi open source, rilasciato in licenza \emph{GNU GPLv2}.
        

        Esso è gestito e sviluppato dalla comunità di sviluppatori e piloti, in collaborazione con le aziende interessate.
        

        Il modello a sorgente libero consente ad esperti, sviluppatori e piloti di godere di una maggiore sicurezza, stabilità e controllo sul software, nonché allo studente di esaminarne liberamente il funzionamento ed apprenderne i meccanismi interni.

\end{frame}

\begin{frame}
        \frametitle{Modelli in esame}
        Scritto in C++, supporta oltre 20 radiocomandi.
        I modelli utilizzati sono:

        \begin{columns}
                \column{0.5\textwidth}
                \begin{itemize}
                        \item il \emph{Taranis FrSky QX7S}
                \end{itemize}

                \column{0.5\textwidth}
                \begin{figure}
                        \centering
                        \includegraphics<1>[width=.8\textwidth]{./qx7s.png}
                \end{figure}

        \end{columns}
\end{frame}

\begin{frame}
        \frametitle{Modelli in esame}
        Scritto in C++, supporta oltre 20 radiocomandi.
        I modelli utilizzati sono:

        \begin{columns}
                \column{0.5\textwidth}
                \begin{itemize}
                        \item il \emph{RadioMaster TX16S}
                \end{itemize}

                \column{0.5\textwidth}
                \begin{figure}
                        \centering
                        \includegraphics<1>[width=.8\textwidth]{./tx16s.png}
                \end{figure}
        \end{columns}
\end{frame}


\section{Materiali e Metodi}

\begin{frame}
        \frametitle{Funzionamento base del firmware}
        Il firmware OpenTX è composto da due distinti applicativi:
        
        \begin{itemize}
                \item il \emph{System Firmware}, responsabile della gestione degli input, della lettura delle impostazioni, della produzione di immagini su schermo
        
                \item il \emph{Transmitter Firmware}, si occupa della trasmissione radio
        
        \end{itemize}
        
        La memorizzazione avviene attraverso una memoria a stato solido.

        Lo schermo consente all'utente la configurazione delle impostazioni, la lettura della telemetria (RSSI, GPS, batteria residua del radiocomando).
        
        Ogni telecomando è in grado di riprodurre annunci vocali.

\end{frame}

\begin{frame}
        \frametitle{Mixer}
        
        I mixer consentono di elaborare uno o più input mediante coefficienti, offset, o funzioni più complesse, e direzionarlo in uno o più canali d'uscita (fino a 32).
        

        Esistono più tipi di mixer:
        
        \begin{description}
                \item[peso ed offset] l'output è ottenuto mediante la formula $output = sorgente \cdot peso + offset$
        
                \item[diff] riduce l'effetto di un input in una direzione e l'amplifica nella direzione opposta
        
                \item[function] applica una funzione matematica o disuguaglianza
        
                \item[expo] applica un andamento esponenziale
        
                \item[curve] applica una curva personalizzata (fino a 17 punti selezionabili)
        \end{description}
\end{frame}

\begin{frame}
        \frametitle{Supporto OpenTX per Lua}

        OpenTX supporta gli script in Lua dalla versione 2.0. 
        

        Il firmware consente l'esecuzione di script con regole e struttura precisa.
        

        Esistono due categorie principali di script:
        
        \begin{itemize}
                \item gli script monouso
        
                \item gli script persistenti
        
        \end{itemize}

        I primi sono caricati all'avvio del firmware o del modello in memoria, i secondi vengono evocati una singola volta ed interrompono gli altri script.
\end{frame}

\begin{frame}
        \frametitle{Script persistenti}
        Sono caricati all'avvio, ed eseguiti fino allo spegnimento del radiocomando o alla selezione di un nuovo modello.
        

        È possibile eseguirne fino ad un massimo di 7 contemporaneamente.
        

        Vi sono più tipi di script persistenti:
        
        \begin{description}
                \item[script di modello] svolgono funzioni analoghe ai mixes, con possibilità arbitrarie di elaborazione
        
                \item[script di funzione] del tutto simili agli script di modello, attivabili con un pulsante o finché una determinata condizione risulta vera
        
                \item[script di telemetria] presentano dati e immagini al pilota, e possono riprodurre annunci vocali
        
                \item[widget] hanno funzionalità paragonabili agli script di telemetria. Sono supportati dai modelli più avanzati, è possibile personalizzarne dimensione e colori da radiocomando
        \end{description}

\end{frame}

\begin{frame}
        \frametitle{Script monouso}
        Gli script monouso sono evocati dal menù del radiocomando, o da una specifica funzione.\newline
        
        Interrompono tutti gli altri script, la loro esecuzione è ripristinata al termine dello script monouso.\newline
        
        Sono l'ideale per realizzare programmi come ad esempio i wizard di installazione.\newline
\end{frame}


\begin{frame}
        \frametitle{Script di telemetria}
        Gli script di telemetria seguono una precisa struttura, e vi sono 3 tipi di funzioni principali:
        
        \begin{description}
                \item[funzione \texttt{init()}] viene eseguita all'avvio dello script (opzionale)
                        
                \item[funzione \texttt{run()}] viene eseguita periodicamente, quando la schermata di telemetria è visibile
                        
                \item[funzione \texttt{background()}] viene eseguita periodicamente a prescindere dalla schermata mostrata
                        
        \end{description}
        Uno statement di \texttt{return} a fine script collega le definizioni delle funzioni, che possono avere un nome diverso, con le funzioni principali richieste da OpenTX.
\end{frame}

\begin{frame}
        \frametitle{Widget}
        I widget sono l'evoluzione degli script di telemetria.\newline 
        
        Nella prima parte:
        
        \begin{description}
                \item[stringa \texttt{nome}] il nome del widget, massimo 10 caratteri
                        
                \item[table \texttt{options}] contiene le opzioni configurabili dal pilota tramite interfaccia grafica
                        
                \item[funzione \texttt{create()}] viene chiamata all'avvio del widget
                        
                \item[funzione \texttt{update()}] viene chiamata quando si verifica una modifica alle opzioni
        \end{description}
\end{frame}

\begin{frame}
        \frametitle{Widget}
        Nella tabella delle opzioni possono essere presenti i seguenti elementi:
        
        \begin{description}
                \item[\texttt{VALUE}] valore numerico costante
                        
                \item[\texttt{SOURCE}] sorgente di input da cui prelevare il valore
                        
                \item[\texttt{BOOL}] variabile booleana costante
                        
                \item[\texttt{COLOR}] colore espresso in formato RGB565
                        
        \end{description}
        Essi sono modificabili dall'utente da radiocomando, mediante le opzioni del widget.\newline
\end{frame}

\begin{frame}
        \frametitle{Widget}
        Le funzioni principali dei widget sono:
        
        \begin{description}
                \item[funzione \texttt{refresh()}] eseguita periodicamente quando lo schermo è visibile
                        
                \item[funzione \texttt{background()}] eseguita periodicamente quando lo schermo non è visibile
                        
        \end{description}
        Anche in questo caso uno statement di \texttt{return} collega le definizioni delle funzioni con le funzioni principali previste da OpenTX.
\end{frame}

\begin{frame}
        \frametitle{Funzioni \texttt{lcd}}
        Si occupano della stampa su schermo:
        
        \begin{itemize}
                \item \texttt{lcd.clear()} permette di azzerare la schermata, obbligatorio negli script di telemetria
                        
                \item la stampa avviene con \texttt{lcd.drawText()} oppure \texttt{lcd.drawTimer()}
                        
                \item \texttt{lcd.setColor()} permette di impostare un colore personalizzato al testo
                        
        \end{itemize}
        Lo schermo del \emph{RadioMaster} supporta una modalità a colori con risoluzione $480 \times 272$, mentre lo schermo del \emph{Taranis} supporta soltanto la modalità bianco e nero con risoluzione $128 \times 64$.\newline
\end{frame}

\begin{frame}
        \frametitle{OpenTX Companion}
        \emph{OpenTX Companion} è un software messo a disposizione dagli sviluppatori di OpenTX per la configurazione del radiocomando interamente da computer desktop.\newline
        Esso consente:
        
        \begin{itemize}
                \item il caricamento del firmware nel radiocomando e la sincronizzazione della scheda SD
                        
                \item la configurazione del radiocomando
                        
                \item la creazione e configurazione dei modelli
                        
                \item l'esecuzione del backup delle impostazioni
        \end{itemize}
\end{frame}

\begin{frame}
        \frametitle{OpenTX Simulator}
        \emph{OpenTX Simulator} è il secondo software a disposizione degli sviluppatori. Esso è in grado di condurre accurate simulazioni del radiocomando con un alto grado di fedeltà. \newline
        
        Sono supportati oltre $20$ radiocomandi differenti.
\end{frame}

\begin{frame}
        \frametitle{OpenTX Simulator}
        Aspetto della simulazione del radiocomando \emph{RadioMaster TX16S}
        \begin{figure}
                \centering
                \includegraphics<1>[width=.8\textwidth]{../Pictures/opentx-sim-first.png}
        \end{figure}
\end{frame}

\begin{frame}
        \frametitle{OpenTX Simulator}
        Aspetto della simulazione del radiocomando \emph{Taranis QX7S}
        \begin{figure}
                \centering
                \includegraphics<1>[width=.8\textwidth]{../Pictures/opentx-sim-qx7s.png}
        \end{figure}
\end{frame}

\section{Risultati e Discussione}

\begin{frame}
        \frametitle{Risultati e Discussione}
        In questa tesi sperimentale sono stati realizzati due script Lua:
        
        \begin{itemize}
                \item lo script per il cronometro vocale, \texttt{TmrCnt}
        
                \item lo script per il conto alla rovescia vocale, \texttt{CntDwn}
        \end{itemize}
        
        Ciascuno script è stato realizzato in due versioni, una per radiocomando in esame.

\end{frame}

\begin{frame}
        \frametitle{\texttt{TmrCnt}}
        Lo script \texttt{TmrCnt} realizza un timer su schermo, con lo scopo di tener conto del tempo trascorso dall'inizio della fase di volo.\newline
        
        Oltre che visualizzare il tempo nella schermata del radiocomando, le informazioni del timer sono riprodotte da annunci vocali.\newline
        
        Questo approccio consente al pilota di mantenere il contatto visivo con il drone, senza la necessità di dover visualizzare l'informazione sul radiocomando.
        
        Lo script fa uso di un timer interno al radiocomando per il conteggio del tempo trascorso.
\end{frame}

\begin{frame}
        \frametitle{\texttt{TmrCnt}}
        Le opzioni dello script presentate al pilota sono:
        
        \begin{description}
                \item[Timer] il timer scelto, con possibili valori $1$, $2$ e $3$
        
                \item[Alert] il numero di secondi che intercorrono fra ciascun annuncio vocale
        
                \item[Color] il colore del testo (per i radiocomandi che supportano i widget)
        
                \item[Invers] flag booleano per l'inversione dei colori (negli schermi in bianco e nero)
        \end{description}
        
        Le due funzioni principali, \texttt{refresh()} e \texttt{background()} vedono i calcoli suddivisi in ulteriori funzioni, \texttt{playTimer()} e la funzione responsabile della stampa della telemetria, \texttt{drawTelemetry()}.\newline
\end{frame}

\begin{frame}
        \frametitle{\texttt{playTimer()}}
        La funzione fa uso di un flag booleano, inizialmente al valore \texttt{false}, per il controllo della riproduzione dell'annuncio vocale.\newline
        
        A seguito della riproduzione dell'annuncio, ulteriori riproduzioni indesiderate sono evitate dal flag al valore \texttt{true}.
        
        Quando si verificano le condizioni idonee, il flag è riportato al valore \texttt{false}.
        
        La lettura del numero avviene mediante la funzione interna \texttt{playDuration()}.
\end{frame}



\begin{frame}
        \frametitle{\texttt{drawTelemetry()}}
        La stampa è realizzata con le seguenti modalità:
        
        \begin{itemize}
                \item tenendo conto delle possibili dimensioni dei widget nella versione per telecomandi che supportano i widget
                        
                \item sfruttando in ogni caso l'intero spazio disponibile, anche per gli schermi con caratteristiche inferiori e non in grado di supportare i widget
                        
                \item per una migliore ottimizzazione, la funzione di stampa è chiamata solo ed esclusivamente tramite la funzione \texttt{refresh()}, attiva nel caso sia visibile la schermata di telemetria
        \end{itemize}
\end{frame}

\begin{frame}
        \frametitle{Aspetto di \texttt{TmrCnt}}
        Lo script si presenta infine con il seguente aspetto (versione per il \emph{TX16S}):
        \begin{figure}
                \centering
                \includegraphics<1>[width=.8\textwidth]{../Pictures/tmrcnt-tx16s.png}
        \end{figure}
\end{frame}

\begin{frame}
        \frametitle{Aspetto di \texttt{TmrCnt}}
        Lo script si presenta infine con il seguente aspetto (versione per il \emph{QX7S}):
        \begin{figure}
                \centering
                \includegraphics<1>[width=.8\textwidth]{../Pictures/tmrcnt-qx7s.png}
        \end{figure}
\end{frame}

\begin{frame}
        \frametitle{\texttt{CntDwn}}
        Lo script \texttt{CntDwn} realizza un conto alla rovescia a partire da un valore impostabile dall'utente (valore predefinito $180$ secondi), con precisi annunci vocali.\newline
        
        Gli annunci vocali vengono riprodotti ogni minuto, e negli ultimi $30$, $15$ secondi, ed infine da $10$ secondi fino allo scadere del tempo residuo. Gli annunci vocali consentono al pilota di non interrompere il contatto visivo con il drone.\newline
        
        La struttura dello script è del tutto simile alla precedente, eccezion fatta per la funzione di riproduzione degli annunci, maggiormente complessa.
\end{frame}

\begin{frame}
        \frametitle{\texttt{playTime()}}
        La funzione dedicata alla riproduzione degli annunci è stata progettata per riprodurre annunci a determinati secondi, e fa ancora uso di un flag booleano di controllo.\newline
        
        La principale differenza è che a causa della maggiore complessità il ripristino del flag avviene in una funzione esterna, \texttt{resetPlayedFlag()}.\newline
        
        Tale funzione si occupa di valutare i casi specifici per cui il flag di controllo debba essere ripristinato al valore \texttt{false}.\newline
        
        Il ripristino infatti avviene sempre al secondo successivo all'annuncio vocale, così da evitare possibili ripetizioni indesiderate.
\end{frame}

\begin{frame}
        \frametitle{Aspetto di \texttt{CntDwn}}
        Lo script si presenta infine con il seguente aspetto (versione per il \emph{TX16S}):
        \begin{figure}
                \centering
                \includegraphics<1>[width=.8\textwidth]{../Pictures/cntdwn-tx16s.png}
        \end{figure}
\end{frame}

\begin{frame}
        \frametitle{Aspetto di \texttt{CntDwn}}
        Lo script si presenta infine con il seguente aspetto (versione per il \emph{QX7S}):
        \begin{figure}
                \centering
                \includegraphics<1>[width=.8\textwidth]{../Pictures/cntdwn-qx7s.png}
        \end{figure}
\end{frame}

\section{Conclusioni}

\begin{frame}
        \frametitle{Conclusioni}
        Lo studio della documentazione, la scrittura degli script e l'analisi del loro comportamento tramite il simulatore è potuta per intero avvenire con mezzi esclusivamente software.\newline
        
        Ogni script è stato ugualmente testato su radiocomandi reali, ottenendo il medesimo comportamento riscontrato durante la fase di test tramite simulatore.\newline
        
        In definitiva, è stato possibile creare due script di telemetria, in grado di supportare radiocomandi dalle caratteristiche assai differenti e con numerose possibilità di espansione futura.
\end{frame}

\section{Sviluppi Futuri}
\begin{frame}
        \frametitle{Sviluppi Futuri}
        Alcuni spunti per la realizzazione di script futuri potrebbero essere:
        
        \begin{itemize}
                \item ottenimento ed elaborazione di ulteriori sorgenti, fra cui
                        
                        \begin{description}
                                        \item[RSSI] intensità del segnale in ricezione
                                                
                                        \item[GPS] coordinate GPS ed indicazione dell'ultima posizione nota in caso di segnale assente
                                                
                                        \item[RxBt] tensione della batteria del modello
                                                
                        \end{description}
                \item produzione di avvisi critici (scarso RSSI, raggiungimento di una determinata area o coordinata GPS, tensione di batteria del modello minore di una certa soglia)
                        
                \item controllo diretto del drone, sempre prestando attenzione alla sicurezza e ad eventualità criticità in tale ambito
        \end{itemize}
\end{frame}


\begin{frame}
        \begin{columns}
                \begin{column}{0.75\textwidth}
                        \begin{center}
                                \font\endfont = cmss10 at 7.00mm
                                \color{uniudBrown}
                                \endfont 
                                GRAZIE PER L'ATTENZIONE
                                \baselineskip 20.0mm
                        \end{center}    

                \end{column}
        \end{columns}
\end{frame}


%\AtBeginSection



\end{document}
